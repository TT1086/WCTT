\section{Trabajo previo}
\noindent
La interacción humano-computador se han diversificado y extendido ampliamente debido al cada vez mayor acceso a las TI ya no sólo a través de un ordenador personal sino también a través del uso cada vez más extendido de dispositivos móviles y el cada vez mayor número de usuarios en redes sociales tanto en navegadores web como en aplicaciones de dispositivos móviles.
\par
Cada vez se da un mayor contenido adaptativo de dichas aplicaciones y entornos web pero enfocado principalmente a datos relacionados con el historial de navegación, comportamiento en redes sociales y otros datos en nada relacionados con el estado emocional de este. Al aplicar los principios del cómputo afectivo y enfocándolos en la interacción humano computador podemos crear ambientes más enriquecedores y usables que se adapten y reaccionen ante el estado emocional del usuario.
\par

    \subsection{Trábajos Académicos}
		\begin{table}[H]
			\centering
				
				\begin{tabularx}{\textwidth}{|p{30mm}|p{25mm}|X|}


					\hline
					Título & Institución & Resumen\\
					\hline
					\hline
					{\bfseries Software inteligente basado en cómputo afectivo} -Trabajo Posgrado- 
					& Escuela Superior de Cómputo - IPN
					& Es un software apto para identificar emociones originadas por estrés, y capaz de restablecer o retomar el equilibrio afectivo orientado a un programa con capacidad para tomar decisiones ante circunstancias catastróficas, como los desastres naturales.
					\\
					\hline
					{\bfseries Interfaces afectivas en el contexto de bibliotecas digitales} -Tesis- 
					& Universidad de las Américas Puebla
					& Este trabajo se enfocó en tomar la idea de Referencia Virtual, sólo que con una mejora característica, el conjunto de emociones presentadas en alguna situación o circunstancia; generando así el sistema Referencia Virtual Afectiva 2.0.
					\\
					\hline
					{\bfseries Emotive Alert: Detección de estados afectivos en correos de voz}
					& Massachussets Institute of Technology
					& Se propuso el sistema “alerta emotiva”, el cual es capaz de detectar una serie de emociones provenientes de un mensaje de voz, informándole así al usuario el nivel de importancia de los mensajes recibidos.
					\\
					\hline
					{\bfseries Monitoreo de emociones aplicadas a terapias basadas en juegos y lógica difusa para adultos mayores}
					& Instituto Tecnológico de León
					& El sistema trata de monitorear el comportamiento de personas de la tercera edad para detectar enfermedades como el Alzheimer y otras relacionadas con la demencia.
					\\
					\hline
					{\bfseries Diseño e implementación de cómputo afectivo para el reconocimiento y generación de comportamientos en un robot.} -Trabajo posgrado
					& Escuela superior de Cómputo - IPN
					& Robot que reacciona de acuerdo al estado emocional actual del usuario desarrollado en posgrado de ESCOM.
					\\
					\hline
				\end{tabularx}
				
			\caption{Trabajo previo : Trabajos académicos}
		\end{table}

    \subsection{Sistemas comerciales}

			\begin{table}[H]
			\centering
				
				\begin{tabularx}{\textwidth}{|p{30mm}|p{25mm}|X|}


					\hline
					Título & Autoría & Descripción\\
					\hline
					\hline
					{\bfseries AFFDEX} -Aplicación- 
					& Affectiva
					& Esta aplicación mide científicamente las respuestas emocionales de manera rentable y a escala. Sin un equipo especial ni requerimientos extras de procesamiento.
					\\
					\hline
					{\bfseries Emotient WEB API} -Aplicación- 
					& Emotient
					& Esta API permite integrar la tecnología de Emotient Analytics para utilizar su motor de reconocimiento facial enfocándose en detectar emociones.
					\\
					\hline
						
				\end{tabularx}
				
			\caption{Trabajo previo : Sistemas comerciales}
		\end{table}
