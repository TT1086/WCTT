\section{Contexto de trabajo}
\noindent En la actualidad se está realizando un análisis más profundo a las técnicas de cómputo afectivo, para así resolver las necesidades del ser humano mediante la interacción humano-máquina.
\par
El Cómputo Afectivo (\textit{Affective Computing}) es una disciplina de la Inteligencia Artificial que intenta desarrollar métodos computacionales orientados a reconocer, detectar, procesar, interpretar e incluso provocar emociones humanas y generar emociones sintéticas.
\par
Esta disciplina surge frente a la necesidad de optimizar la interacción entre personas y computadoras, pero también incluye la investigación de los procesos inteligentes.
\par
La fundadora de esta área de investigación es Rosalind Piccard, investigadora del M.I.T. (Massachussets Institute of Technology), quien menciona que Las emociones forman una parte muy importante en los procesos inteligentes, ejemplo de ello es la toma de decisiones, por lo que se deben tener en cuenta los procesos emocionales y la forma en que estos participan en la inteligencia.”'\cite{NOTA}
\par
El principal objetivo del Cómputo Afectivo es desarrollar la mejor interacción humano-computadora posible. Dicha interacción se logra mediante la solución de dos problemáticas:
\begin{enumerate}
\item El reconocimiento de emociones (expresiones emotivas) humanas por parte de 	una computadora: cuyo objetivo es captar aquellos signos relacionados con la expresión de emociones y lograr interpretar estados emocionales en función de dichos signos.
\item La simulación (o generación) de estados y expresiones emocionales con computadoras: la cual intenta que las computadoras puedan simular procesos emocionales con base en ciertos modelos.
\end{enumerate}
\par
En México se han realizado algunos avances en la Escuela Superior de Cómputo, por lo que actualmente basados en técnicas de reconocimiento de patrones y tratamiento de
imágenes, podemos reconocer distintos estados emocionales, empleando plantillas que se relacionan con diferentes expresiones faciales. Mediante procesamiento de voz se puede identificar el estado emocional de una persona con un grado de certeza bastante aceptable.
\par
Actualmente se está trabajando en desarrollar sistemas como juegos que interactúan dependiendo de la situación o estado emocional del usuario, sistemas de aprendizaje y también para reconocer enfermedades a través de las emociones. En el mundo hay distintos proyectos de investigación relacionados a esta rama computacional.
\par