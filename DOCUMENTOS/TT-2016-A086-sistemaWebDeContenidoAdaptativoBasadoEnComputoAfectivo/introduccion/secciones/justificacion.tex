\section{Justificación}
\noindent
La relevancia de este Trabajo Terminal se centra en Cómputo Afectivo, que como área de investigación es reciente en México. Dentro de sus múltiples aplicaciones podemos encontrar todas aquellas enfocadas en una alta interacción humano-computadora; tales como atención al cliente, marketing, educación, sistemas tutoriales, entre otras, cada una de las cuales repercutirá en diferentes beneficiarios potenciales.
\par
El desarrollo de este sistema involucra diferentes áreas de conocimiento como: Ingeniería de Software, Reconocimiento de Patrones, Desarrollo Web, Bases de Datos, Análisis Estadístico de la Información y dependiendo del área de aplicación Psicología, Pedagogía o Mercadotecnia. Todos estos conocimientos integrados para dar respuesta a una necesidad actual.
\par
La originalidad de este Trabajo Terminal radica en ser de los primeros que se basará en el reconocimiento facial para la detección de emociones, dado que los trabajos elaborados anteriormente se han basado en otras técnicas como reconocimiento de voz, patrones de movimiento, análisis de imágenes y texto.

  