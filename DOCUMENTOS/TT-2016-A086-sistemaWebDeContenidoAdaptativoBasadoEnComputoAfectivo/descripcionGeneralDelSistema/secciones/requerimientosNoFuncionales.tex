\section{Requerimientos no funcionales}

    

    \subsection{RNFN1 Compatibilidad multiplataforma}

	   El acceso al sistema se podrá realizar desde plataformas Linux, Windows y Mac.
	   
	  RNFG2. Compatibilidad multiplataforma

\subsection{RNFN2 Usabilidad}
El sistema debe proporcionar mensajes de error que sean informativos y orientados a usuario
final. También proporcionará mensajes de ayuda en los principales elementos de navegación del sistema.

\subsection{RNF3 Lenguajes de programación}
Se utilizará Java como lenguaje de programación para el desarrollo de la aplicación.
\subsection{RNF4 Desarrollo dinámico web}
Se utilizará JSF a través de JSP para acelerar el desarrollo del contenido web de la aplicación.
\subsection{RNFN5 Sistema Gestor de Bases de Datos}
MySQL será el SGBD a utilizar debido a la fácil interacción con la herramienta de desarrollo
en el mapeo de objetos en datos y viceversa.
\subsection{RNFN6 Contenedor web}
Glassfish es el contenedor web elegido para que resida la aplicación.
\subsection{RNFN6 Navegador web}
El sistema será compatible con los navegadores Mozilla Firefox y Google Chrome cubriendo
ambos al 73.52\% \cite{GLOBALStat} de los usuarios mexicanos de navegadores web (68.44\% el navegador
de Google y el restante 5.08 el de Mozilla) cubriendo a gran parte de la población .
\subsection{RNFN7 Usabilidad - Tiempo aprendizaje}
El tiempo de aprendizaje del sistema por un usuario deberá ser menor a 30 minutos.

